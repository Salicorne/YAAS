\documentclass[a4paper, 11pt]{report}
\usepackage[utf8]{inputenc}
\usepackage{graphicx}

\title{YAAS Project : Report}
\author{Timothée RABOEUF - 74764}
\date{date} 
    
\begin{document} 
\maketitle
\tableofcontents
    
\chapter{Introduction}

    \section{Implemented requirements}

    The following list of requirements has been implemented : 
    \begin{itemize}
        \item UC1: create user
        \item UC2: edit user
        \item UC3: create auction
        \item UC4: edit auction description
        \item UC6: bid
        \item UC5: Browse \& Search
        \item UC7: ban auction
        \item UC8: resolve auction
        \item UC9: language switching
        \item UC10: concurrency
        %\item UC11: currency exchange
        \item WS1: Browse \& Search API
        \item WS2: Bid api
        %\item OP2: soft deadlines for bidding
        %\item OP3: store language preference
        %\item OP1: send seller auction link
        %\item TR1: data generation program
        %\item TR2

    \end{itemize}

    \section{Packages used}
    This YAAS application relies on the following packages :
    \begin{itemize}
        \item Django (2.1.1)
        \item django-crispy-forms (1.7.2)
        \item django-filter (2.0.0) 
        \item djangorestframework (3.8.2)
        \item Markdown (3.0.1)
        \item requests (2.20.0)
        
    \end{itemize}

\chapter{Development strategies}

    \section{Session management}

    \section{Confirmation form (UC3)}

    UC3 states that the user should be asked for a confirmation before creating a new auction. I implemented it using the rendering of a confirmation form, containing all auction informations in hidden field. The first auction creation form is validated server-side : if all informations are correct, the confirmation form is shown with these informations. Only when this confirmation form is submited the new auction is created and saved in database : thus if the user cancels the process or never submits the form the auction is not saved. This behaviour is described in the file \texttt{Auction/views.py}, in the functions \texttt{AuctionEditView::get}, \texttt{AuctionEditView::post} and \texttt{auctionConfirm}. 

    \section{Automatic bid resolution}

    For the bids resolution, the problematic is to allow resolution without any web request being made. We can't use classic Django views, and need to find another method. To begin I had a look at the Django module \texttt{django\_cron} but all it does is exposing a django command allowing to run jobs through \texttt{manage.py}, but still relies on an external tool (such as Unix crontab) to run this command. The module \texttt{django-extensions} has a jobs scheduling system, but it does not run jobs automatically either. I ended up manually launching a new thread, that will check all active (ie not banned and not resolved) auctions every minute and resolve them if needed. 
        
    \section{Concurrency management}

    \section{REST API}

    The REST API is implemented using the \textit{Django REST Framework}. It follows main REST principles : it relies on HTTP verbs (GET, POST), and is \textit{stateless} : there is no state kept on the server, and each request is independant. The code is shared between the regular web app and the API : for a single action there is a function for the actual action, and two other englobing functions used to call this main function from the regular views and from the API. The naming convention is as follows :

    \begin{itemize}
        \item \texttt{get\_function} : The actual code logic
        \item \texttt{api\_function} : Used to call the function from the API
        \item \texttt{function} : Used to call the function from the web app.
    \end{itemize}

    \section{Functional tests}

    \section{Language switching management}

    \section{Data generation}

\end{document}